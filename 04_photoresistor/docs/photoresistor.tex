\documentclass[a4paper]{proc}

%% PACKAGES
\usepackage[european, siunitx]{circuitikz}
\usepackage[utf8]{inputenc}
\usepackage{hyperref}
\usepackage{float}

%% DOCUMENT
\begin{document}
	%% Title and base informations
	\author{Naoki Pross 2B}
	\title{Photoresistor Project}
	\markboth{SAM Bellinzona}{}
	\maketitle
	
	\begin{abstract}
		Il progetto consiste nel visualizzare i dati letti dal fotoresistore con
		un Arduino megaADK (ATmega2560), evitando nel limite del possibile di
		utilizzare le funzioni introdotte da \texttt{Arduino.h}.\\\\
		Questo progetto e` un estensione di un altra ricerca fatta nel tempo
		libero personale sul funzionamento di un display LCD standard.
	\end{abstract}
	
	\section{Elettronica}
	Il metodo scelto per visualizzare i dati e` attraverso un display LCD, percio`
	ho avuto la necessita` di effettuare delle ricerche su come controllare un display
	LCD standard. (vedi \texttt{NaoPross/samb2\_hw\_and\_sw/03\_LCD\_display} su GitHub).
	Quindi sono necessarie due porte e un pin analogico. Per il montaggio sono stati 
	scelti i seguenti pin/porte.
	
	\begin{table}[h]
		\caption{Pin e porte scelte per il montaggio}
		\centering \begin{tabular}{ l l }
			\hline \\
			\texttt{PORT A} & Ouput Data Bus connected to the LCD \\
			\texttt{PORT C} & \texttt{RS / RW / E} Control pins for the LCD\\
			\texttt{PIN A0} & Analog Pin to read the voltage\\
			\\ \hline
		\end{tabular}
	\end{table}
	
	Lo schema elettrico del circuito si riduce ad una cosa molto semplice,
	un potenziometro in serie ad una fotoresistenza, per poter regolare la sensibilita`
	della misurazione.
	
	\begin{figure}[H]
		\centering
		\begin{circuitikz}
			% display
			\node[draw,minimum width=4.25cm,minimum height=1.5cm,anchor=south west](lcd) {\texttt{LCD Display}};
			% display CON port & data bus
			\foreach \i in {1,2, ..., 16}
			{
				\draw (.25 * \i, 0) -- (.25 * \i, -1);
			}
			% Arduino
			\draw
				(0, -1) node[draw,minimum width=5cm,minimum height=2cm,anchor=north west](atmega) {\texttt{ATmega2560}}
				(.05, -1.25)		node[scale=.4,anchor=west] {\texttt{+5V GND}}
				(1.25, -1.25)	node[scale=.5,anchor=west] {\texttt{PORTA}}
				(2.5, -1.25)		node[scale=.4,anchor=west] {\texttt{PC0 PC1 PC2}}
				(3.5, -1.25)		node[scale=.4,anchor=west] {\texttt{+5V GND}}
				(4.5, -1.25)		node[scale=.4,anchor=west] {\texttt{+5V}}
				(4.5, -2.75)		node[scale=.4,anchor=west] {\texttt{GND}};
			% USB to PC
			\foreach \i in {1,2, ..., 4}
			{
				\draw (.25 * \i, -3) -- (.25 * \i, -4);
			}
			\draw
				(.6, -4.25)		node[anchor=north] {\texttt{USB Cable}};
			% Photoresistor
			\draw
				(4.7, -1)	to[short]						(4.7, 1)
						 		to[short]						(6, 1)
								to[variable resistor=$R_{var}$] 		(6, -1)
								to[short]						(6, -2)
								to[photoresistor=$R$]			(6, -4)
								to[short]						(4.7, -4)
								to[short]						(4.7, -3);
			\draw
				(6, -2)	to[short, o-]		(5, -2)
				(4.5, -2) node[scale=.4,anchor=west] {\texttt{A0}};
		\end{circuitikz}
		\caption{Schema elettrico del montaggio}
	\end{figure}
	
	\section{Software}
	Il codice si riduce ad una base molto semplice, quindi dei cicli infiniti
	di lettura, codifica del numero letto in ASCII e visualizzazione del testo
	sull'LCD.
	Tutto il codice sorgente si trova al repo
	\texttt{samb\_hw\_and\_sw/07\_photoresistor} su Github.
	
	%Tutta la libreria di comunicazione e` stata scritta nel progetto precedente quindi
	%l'unico problema riscontrato era che il valore letto dal sensore veniva invertito.
	%Quindi se la luminosita` incrementava, il valore letto scendeva. Questo perche`
	%la lettura veniva eseguita sulla tensione del carico $R$ che era un fotoresistore.
	%I fotoresistori hanno la caratteristica di aumentare la resistenza proporzionalmente
	%rispetto alla intensita` di luce a cui sono esposti, quindi incrementando la luminosita`
	%con la legge i Ohm la tensione scendeva.
	
	%$$I = const.$$
	%$$U = R * I$$
\end{document}