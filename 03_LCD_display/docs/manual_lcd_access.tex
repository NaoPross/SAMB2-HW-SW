%% Naoki Pross 2B/SAMB
\documentclass[a4paper]{article}        % choose between 
                                        %  - article
                                        %  - IEEEtran
                                        %  - proc
                                        %  - report 

%% PACKAGES
\usepackage{circuitikz}                 % to draw electric schemes
\usepackage[utf8]{inputenc}             % to write accents
\usepackage{float}                      % required to place figures/.. w [H]
\usepackage{hyperref}                   % to write URL as \url{link}

%% DOCUMENT
\begin{document}
	%% Title and base informations
	\author{Naoki Pross 2B}
	\title{Controllo Manuale di un LCD}
	\markboth{SAM Bellinzona}{}
	\maketitle

    \begin{abstract}
        L'idea consiste nel riscrivere la librearia per il controllo dei
        display LCD data da \texttt{Arduino.h}, partendo dalle informazioni
        fornite per il display di cui dispongo, \texttt{YB1602A}.
    \end{abstract}
	
	\section{Elettronica}
    Il display \`e fornito con un driver \texttt{SunPlus SPLC780C} che
    permette di generare dei caratteri 5x8 sul display. Il componente
    dispone di una porta \texttt{CON1} a 16 pin descritti nel datasheet.
    
    \begin{table}[h]
        \centering
        \begin{tabular}{ r l c l }
            Pin & Simbolo & Valore & Funzione \\
            \hline
                 1 & \texttt{VSS}      & \texttt{GND}  & Massa di alimentazione \\
                 2 & \texttt{VDD}      & \texttt{+5V}  & Alimentazione \\
                 3 & \texttt{$V_0$}    &               & Contrasto \\
                 4 & \texttt{RS}       & \texttt{H / L}& Dati / Comando \\
                 5 & \texttt{RW}       & \texttt{H / L}& Read / Write \\
                 6 & \texttt{E}        & \texttt{H / L}& Clock Stop / Resume \\
            7 - 14 & \texttt{DB0 - DB7}& \texttt{H / L}& Data Bus \\
                15 & \texttt{LEDA}     & \texttt{+5V}  & Alimentazione backlight \\
                16 & \texttt{LEDK}     & \texttt{GND}  & Massa di alimetazione del backlight\\
            \hline
        \end{tabular}
        \caption{Funzione dei pin di un display LCD standard}
    \end{table}

\end{document}
